%        File: !comp!expand("%")!comp!
%     Created: !comp!strftime("%a %b %d %I:00 %p %Y ").substitute(strftime('%Z'), '\<\(\w\)\(\w*\)\>\(\W\|$\)', '\1', 'g')!comp!
% Last Change: !comp!strftime("%a %b %d %I:00 %p %Y ").substitute(strftime('%Z'), '\<\(\w\)\(\w*\)\>\(\W\|$\)', '\1', 'g')!comp!
%
\documentclass[twoside]{article}
\usepackage[tmargin=2.5cm,bmargin=2.5cm,lmargin=2cm,rmargin=2cm]{geometry}
\usepackage{graphicx}
\usepackage{caption}
\usepackage{enumerate}
\usepackage{framed}
\usepackage{amssymb}
\usepackage[svgnames]{xcolor}
\usepackage{comment}
\usepackage[normalem]{ulem}
\usepackage{amsmath}
\usepackage{fancyhdr}
\usepackage{hyperref}
\usepackage{physics}
\usepackage{siunitx}
\usepackage{float}
\usepackage{xcolor}
\pagestyle{fancy}
\fancyhf{}
\lhead{}
\chead{}
\cfoot{\thepage}

\begin{document}

\section{The problem: find expressions for bubble and triangle integrals}
We are interested in two types of integrals, which we call the bubble integral and the triangle integral. The bubble integral has the form,
\begin{align}
\label{eq:bubble}
B(k^2,M_1,M_2) =  \int \frac{d^3q}{\pi^{3/2}}\frac{1}{(q^2 + M_1)( (k-q)^2 + M_2)}
\end{align}
and the triangle integral has the form,
\begin{align}
T(k_1^2, k_2^2, k_3^2, M_1, M_2, M_3) =  \int \frac{d^3q}{\pi^{3/2}}\frac{1}{(q^2 + M_1)( (k_1-q)^2 + M_2)( (k_2 + q)^2 + M_3)},
\end{align}
where $k_1 + k_2 + k_3 = 0$.
To compute these integrals, we use Feynman parameterization and dim reg. When the masses $M_1,M_2$ and $M_3$ are real, we do not
have to worry about branch cuts and analytic continuation of the above integral in the complex plane. However, we will need to be more
careful in the case of complex masses. 

One is allowed to introduce Feynman parameterization even with complex masses but one needs to treat carefully the extra branch cuts to the integrand.

\section{Preliminaries}
\label{sec:prelim}

In this section, we will state some useful ingredients that are helpful in our calculation.

\subsection{Generalized multiplicative formulas for logs and square roots}

Following the paper by t'Hooft and Veltman (1979), we use logs and square roots that have a branch cut in the negative real axis.

The rule for the log of a product is then:
\begin{equation}
\begin{split}
& \log(a b) = \log(a) + \log(b) + \eta(a,b)\,, \\
& \eta(a,b) = 2 \pi i \left[ \theta(- \Im a) \theta(- \Im b) \theta(\Im ab) - \theta(\Im a) \theta(\Im b) \theta(-\Im ab) \right]\,,
\end{split}
\end{equation}
where $\theta$ is the Heaviside step function, and $a$ and $b$ are complex numbers. If $a$ (and/or $b$) is a negative real, one should prescribe $\Im a = i \, \epsilon$, where $\epsilon>0$. Thus in that case, we have $\theta(\Im a)=1$ and $\theta(-\Im a)=0$. For consistency, if $a$ is a positive real, we prescribe $\theta(\Im a)=0$ and $\theta(-\Im a)=1$.

Useful consequences are:
\begin{align}
& \log(a b) = \log(a) + \log(b) \,, \text{ if $\Im a$ and $\Im b$ have different sign,}\\
& \log(\frac{a}{b}) = \log(a) - \log(b) \,, \text{ if $\Im a$ and $\Im b$ have the same sign.}
\end{align}

Defining the square root as $\sqrt{a} \equiv \exp(\log(a)/2)$, we obtain the analogous expressions:
\begin{equation}
\begin{split}
& \sqrt{a b} = s(a,b)\sqrt{a}\sqrt{b}\,, \\
& s(a,b) \equiv \exp{\eta(a,b)/2} =(-1)^{\theta(- \Im a) \theta(- \Im b) \theta(\Im ab)} (-1)^{\theta(\Im a) \theta(\Im b) \theta(-\Im ab)}
\,,
\end{split}
\end{equation}
and specifically:
\begin{align}
& \sqrt{a b} = \sqrt{a}  \sqrt{b} \,, \text{ if $\Im a$ and $\Im b$ have different sign,}\\
& \sqrt{\frac{a}{b}} = \frac{\sqrt{a}}{\sqrt{b}}\,, \text{ if $\Im a$ and $\Im b$ have the same sign.}
\end{align}

As a quick example, let us look at $\sqrt{-z}$.
Applying the product rule in the numerator, we obtain:
\begin{equation}
\sqrt{-z} = s(-1,z)\, i \sqrt{z} =  (-1)^{\theta(\Im 1) \theta(- \Im z)} (-1)^{\theta(\Im -1) \theta(\Im z)} i \sqrt{z}\,,
\end{equation}
and using our prescription, we get $\theta(\Im -1) = 1$ and $\theta(\Im 1) = 0$, and our result simplifies to 
\begin{equation}
\sqrt{-z} = s(-1,z)\, i \sqrt{z} =  (-1)^{\theta(\Im z)} i \sqrt{z}\,,
\end{equation}
which means that $\sqrt{-z} = i\sqrt{z}$ only in the lower complex plane and in the positive real axis, while $\sqrt{-z} = -i\sqrt{z}$ in the upper complex plane and in the negative real axis.


\section{Bubble integral using Schwinger parameters}

Let us derive the Feynman parametrization for the bubble integral using Schwinger parameters. We will use two important (equivalent) identities:
\begin{equation}
\label{eq:schwinger1}
  \frac{i}{A} = \int_0^\infty ds (1 + i\epsilon)\, \exp(i A (1 + i \epsilon)s)\,,
\end{equation}
if $\Im A >0$, and another analogous result
\begin{equation}
\label{eq:schwinger2}
-\frac{i}{A} = \int_0^\infty ds (1 - i\epsilon)\, \exp(-i A(1 - i\epsilon) s)\,,
\end{equation}
if $\Im A <0$. We will constrain the value of $\epsilon$ in both cases to ensure the integrals are convergent. Eqs.~\eqref{eq:schwinger1} and \eqref{eq:schwinger2} are
convergent for $\epsilon = 0$ but it will become important later that $\epsilon$ is positive and non-zero. Let us now consider the case of
$\Im m > 0$. Eq.~\eqref{eq:schwinger1} is convergent if,
\begin{align}
\label{eq:cond1}
  \Re\left[ i(\Re A + i \Im A)(1 + i\epsilon) \right] = -\Im A - \Re A \epsilon< 0.
\end{align}
Taking the form of $A$ to be $q^2 + m$, the conditions on the imaginary part of $A$ become the conditions on the imaginary part of
$m$ since $q$ is real. Eq.~\eqref{eq:cond1} then becomes,
\begin{align}
  -\Im m - (q^2 + \Re m)\epsilon < 0,
\end{align}
which now depends on the sign of $\Re m$. If $\Re m < 0$,
\begin{align}
  \epsilon < \frac{\Im m}{\lvert \Re m \rvert},
\end{align}
where we are free to choose $\epsilon$ to be arbitrarily small but positive. If $\Re m > 0$,
\begin{align}
  \epsilon > -\frac{\Im m}{\Re m},
\end{align}
we are again free to choose $\epsilon_-$ to be arbitrarily small but positive.

Now consider the case for $\Im m < 0$, then Eq.~\eqref{eq:schwinger2} is convergent if,
\begin{align}
  \Im m - (q^2 + \Re m)\epsilon < 0.
\end{align}
The inequality is bounded by the case when $q^2 = 0$, so if $\Re m < 0$,
\begin{align}
  \epsilon < \frac{\lvert \Im m\rvert}{\lvert \Re m\rvert},
\end{align}
and if $\Re m > 0$,
\begin{align}
  \epsilon > \frac{\Im m}{\Re m}.
\end{align}
In both cases, we are still free to take $\epsilon$ to be positive and arbitrarily small.

Looking at the bubble integral, we note that the imaginary parts of each term in the denominator is the same as the imaginary part of the corresponding mass. 

\subsection{Masses with the same imaginary part sign}

Let us then first look at the case where both masses have a positive imaginary part. Using our identities, we obtain straightforwardly:
\begin{equation}
\label{eq:bubble}
B(k^2,M_1,M_2)= - \int_0^\infty ds_1 \int_0^\infty ds_2\,\int \frac{d^3q}{\pi^{3/2}} e^{i(q^2+M_1)s_1}e^{i((k-q)^2+M_2)s_2}\,.
\end{equation}
Simplifying this, we obtain:
\begin{equation}
\begin{split}
B(k^2,M_1,M_2)&= - \int_0^\infty ds_1 \int_0^\infty ds_2\,\int \frac{d^3q}{\pi^{3/2}} (1 + i\epsilon^1)(1 + i\epsilon^2)\\
&\exp{i F_+ (q+ \frac{s_2(1 + i\epsilon^2)}{F_+}k)^2+i\frac{s_1 s_2(1 + i\epsilon^1)(1 + i\epsilon^2)}{F_+}k^2+ i (M_1 s_1(1 + i\epsilon^1) +M_2 s_2(1 + i\epsilon^2))}\,,
\end{split}
\end{equation}
where $F_+= (1 + i\epsilon^1)s_1+(1 + i\epsilon^2)s_2$.

We can then do 2 things that give the same result:
\begin{itemize}
\item We do directly the Gaussian integral in $q$, then do a change of variable $s_1 = \tau x$, $s_2 = \tau (1-x)$ and then integrate in $\tau$, or alternatively
\item We do a change of variable $s_1 = \tau x$, $s_2 = \tau (1-x)$, integrate in $\tau$, and then do the $q$ integral in the standard dim reg way.
\end{itemize} 

Let us pick the first way for concreteness.
We can now set $\epsilon^1=\epsilon^2=\epsilon$, and guarantee that the Gaussian integral converges (because we have $F_+ = s_1 + s_2 + i \tilde{\epsilon}$, where we redefined the $\epsilon$).
After doing the Gaussian $q$ integral, and setting $\epsilon \to 0$ where no poles show up, we get
\begin{equation}
B(k^2,M_1,M_2)= - \int_0^\infty ds_1 \int_0^\infty ds_2\,\frac{1}{(- i(s_1+s_2+ i \epsilon))^{3/2}} \exp{i\frac{s_1 s_2}{s_1+s_2+i\epsilon}k^2+ i (M_1 s_1 +M_2 s_2)}\,,
\end{equation}
where we used the result
\begin{equation}
\label{eq:gauss}
\int d^d q \exp(i a q^2) = \frac{\pi^{d/2}}{(-i a)^{d/2}}\,,
\end{equation}
if $\Im a > 0$.
Now, doing the change of variable described before: $s_1 = \tau x$, $s_2 = \tau (1-x)$, and noting that the Jacobian of the transformation is $\tau$, we get
\begin{equation}
B(k^2,M_1,M_2)= - \frac{1}{(- i)^{3/2}} \int_0^1 dx \int_0^\infty d\tau\, \tau (\tau+i \epsilon)^{-3/2}  \exp{i \frac{\tau^2}{\tau+i \epsilon} x (1-x)k^2+ i \tau (M_1 x +M_2 (1-x))}\,,
\end{equation}
and since the integral is convergent when $\epsilon \to 0$, we get:
\begin{equation}
B(k^2,M_1,M_2)= - \frac{1}{(- i)^{3/2}} \int_0^1 dx \int_0^\infty d\tau\, \tau^{-1/2}  \exp{i \tau x (1-x)k^2+ i \tau (M_1 x +M_2 (1-x))}\,,
\end{equation}
and doing the $\tau$ integral yields the standard Feynman parameter integral:
\begin{align}
B(k^2,M_1,M_2)&= \frac{\Gamma(1/2)}{(- i)^{3/2}} \int_0^1 dx \frac{(-i)^{3/2}}{\sqrt{x (1-x)k^2+ M_1 x +M_2 (1-x)}}\\
& = \sqrt{\pi}\int_0^1 dx \frac{1}{\sqrt{x (1-x)k^2+ M_1 x +M_2 (1-x)}} \,, \label{eq:Bsame}
\end{align}
which is our familiar Feynman integral.  
Note that in this case the square root does not have any branch cut, because its argument always has a positive imaginary part, by hypothesis. We were thus able to find the Feynman parameter integral using Schwinger parameters for this case. For two masses with negative imaginary parts, the exact same steps apply, and we obtain the same result.

Solving this integral yields:
\begin{equation}
\label{eq:Bfinalsame}
B(k^2,M_1,M_2) =\frac{\sqrt{\pi}}{k} \left[ i \log \left(2 \sqrt{x (1-x) + m_1 x + m_2 (1-x)}+i (m_1-m_2-2 x+1)\right)\right]_{x=0}^{x=1} \,,
\end{equation}
where $m_1 = M_1/k^2$ and $m_2 = M_2/k^2$. To use this expression, we need to check if the argument of the $\log$ crosses its branch cut (the negative real axis). If it does, we need to add/subtract $2 \pi i$, depending on the direction of the crossing.

We can finally analyze under what conditions a log branch cut crossing happens. Let us define 
\begin{equation}
A(x, m_1, m_2) \equiv 2 \sqrt{x (1-x) + m_1 x + m_2 (1-x)}+i (m_1-m_2-2 x+1)\,,
\end{equation}
then, we have a branch cut crossing when $A(x, m_1, m_2) = -t$, where $t>0$, for $x \in ]0,1[$
Solving for $x$ yields
\begin{align}
& 2\sqrt{x (1-x) + m_1 x + m_2 (1-x)}+i (m_1-m_2-2 x+1) = -t \\
\Rightarrow\, & 2\sqrt{x (1-x) + m_1 x + m_2 (1-x)} = -t - i (m_1 - m_2 - 2 x +1) \\
\Rightarrow\, & 4 x (1-x) + 4 m_1 x + 4 m_2 (1-x) = t^2 + 2 i t (m_1 - m_2 - 2 x +1) - (m_1 - m_2 - 2 x +1)^2 \\
\Rightarrow\, & \Delta(m_1, m_2) = t^2 + 2 i t (m_1 - m_2 + 1) - 4 i t x  \,,
\end{align}
where
\begin{equation}
\Delta(m_1, m_2) \equiv m_1^2-2 m_1 m_2+2 m_1+m_2^2+2 m_2+1\,.
\end{equation}
which gives two constraints: one for the real part, and another for the imaginary part. The real part equation gives us directly $t$
\begin{equation}
t_\pm = \Im(m_1)-\Im(m_2)\pm\sqrt{\Delta(\Re(m_1), \Re(m_2))}\,,
\end{equation}
and the imaginary part equation gives us the two possible solutions for $x$ corresponding to the two solutions for $t$:
\begin{align}
& x_1 = \frac{- 2 \Im(m_2) + (\Re(m_1)-\Re(m_2)+1) \sqrt{\Delta(\Re(m_1), \Re(m_2))}}{2 \left(\Im(m_1)-\Im(m_2)+\sqrt{\Delta(\Re(m_1), \Re(m_2))}\right)} \,,\\
& x_2 = \frac{- 2 \Im(m_2) - (\Re(m_1)-\Re(m_2)+1) \sqrt{\Delta(\Re(m_1), \Re(m_2))}}{2 \left(\Im(m_1)-\Im(m_2)-\sqrt{\Delta(\Re(m_1), \Re(m_2))}\right)}\,.
\end{align}
However, only one of these solutions, $x_2$ corresponds to a real number. This means that there is at most one branch cut crossing of the $\log$ in Eq.~\eqref{eq:Bfinalsame}. In fact, there is a branch cut if $x_2 \in ]0,1[$ and $t_- > 0$.






\subsection{Masses with opposite sign of the imaginary part}

Let us focus now on the case where the two masses have a different sign in the imaginary part. For concreteness, we assume $\Im M_1 > 0$ and $\Im M_2 < 0$.

In this case we need non-zero $\epsilon$ insertions with Eqs.~\eqref{eq:schwinger1} and \eqref{eq:schwinger2} since we will develop poles without them. Eq.~\eqref{eq:bubble} then becomes,
\begin{align}
 B(k^2, M_1, M_2) &= \int_0^\infty ds_1 ds_2\frac{d^3q}{\pi^{3/2}}(1 + i\epsilon^1)(1 - i\epsilon^2)\exp\Big( i S \times \left( q + \frac{k(1 - i\epsilon^2)s_2}{S} \right)^2 - i\frac{k^2(1+i\epsilon^1) s_1 (1-i\epsilon^2) s_2}{S} \nonumber \\
&+ i\left( (1 + i\epsilon^1)s_1 M_1 - (k^2 + M_2)(1 - i\epsilon^2)s_2 \right) \Big),
\end{align}
where $S = (1 + i\epsilon^1)s_1 - (1-i\epsilon^2)s_2 = s_1 - s_2 + i \tilde{\epsilon}$. The $q$ integral will be convergent for $\epsilon^1 > 0$ and $\epsilon^2 > 0$. Doing the momentum integral and taking $\epsilon \rightarrow 0$ where possible, we obtain,
\begin{align}
  B(k^2, M_1,M_2) = \frac{1}{(-i)^{3/2}}\int ds_1ds_2\frac{1}{S^{3/2}}e^{iI}
\end{align}
where $I = M_1 s_1 - M_2 s_2 - k^2 s_1s_2/S$. Now we make the
change of variables, $s_1 = \tau x$ and $s_2 = \tau(1-x)$ and also take $\epsilon^1 = \epsilon^2 = \epsilon$,
\begin{align}
  B(k^2,M_1,M_2) = \frac{1}{-i^{3/2}}\int^1_0dx\int^{\infty}_0d\tau \frac{\tau^{-1/2}}{(2x-1 + i\epsilon)^{3/2}}\exp\left( i\tau\left(
      M_1 x - M_2(1-x) - \frac{k^2x(1-x)}{2x-1 + i\epsilon}
  \right)\right)
\end{align}
The $\tau$ integration then gives,
\begin{align}
 B(k^2,M_1,M_2) =   -\sqrt{\pi}\int^1_0dx \frac{1}{(2x - 1 + i\epsilon)^{3/2}}\frac{1}{\sqrt{-\frac{k^2x(1-x)}{2x-1+ i\epsilon} + M_1x - M_2(1-x)}}.
\end{align}
We evaluate this integral by taking the principal value using the identity,
\begin{align}
  \frac{1}{X - X_0+ i\epsilon} = P.V.\frac{1}{X-X_0} - i\pi \delta  (X- X_0) .
\end{align}
We obtain,
\begin{align}
\label{eq:PVintegral}
 B(k^2,M_1,M_2) = -\sqrt{\pi}\left(P.V. \int^1_0dx \frac{1}{2x - 1}\frac{1}{\sqrt{2x - 1}\sqrt{-\frac{k^2x(1-x)}{2x-1} + M_1x - M_2(1-x)}} -\frac{i \pi}{2} (-2 i) \right).
\end{align}
The principal value term can be evaluated by using $\arctan$ and carefully choosing a specific Riemann sheet.
Specifically, one obtains:
\begin{equation}
\begin{split}
B(k^2, M_1, M_2) = -\frac{\sqrt{\pi}}{k}&\left(-\frac{2 i \sqrt{- y_1} \sqrt{- y_2}}{\sqrt{2 y_1-1} \sqrt{1- 2 y_2} \sqrt{-(1 + 2(m_1 + m_2)) y_1 y_2}} \left[ \arctan \left(\frac{\sqrt{1-2 y_2} \sqrt{x -y_1}}{\sqrt{2 y_1-1} \sqrt{x - y_2}}\right)\right]_{x=0}^{x=1}\right. \\
&\left. - \pi \right)\,,
\end{split}
\end{equation}
where $y_1$ and $y_2$ are the roots of the second-degree polynomial $P(x)$ given by
\begin{equation}
P(x) = (x -1) x +(2 x -1)\, x \, m_1 +\left(2 x ^2-3 x +1\right) m_2\,,
\end{equation}
giving for $y_1$ and $y_2$:
\begin{align}
y_1 &= \frac{-\sqrt{m_1^2-2 m_1 m_2 + 2 m_1+m_2^2+2 m_2+1}+m_1+3 m_2+1}{2 (2 m_1+2 m_2+1)} \\
y_2 &= \frac{\sqrt{m_1^2-2 m_1 m_2+2 m_1+m_2^2+2 m_2+1}+m_1+3 m_2+1}{2 (2 m_1+2 m_2+1)}\,.
\end{align}
Again, the crossings of the $\arctan$ branch cuts need to be taken into account carefully (by adding or subtracting $\pi$, or $\pi/2$ in the case the crossing happens exactly at $i$ or $-i$, for example when $x=1/2$).
This expression agrees with the numerical momentum integral.

We can also take the expression in Eq.~\eqref{eq:PVintegral} and do a change of variable to put it in a more familiar form.
Indeed, by introducing $\hat{x} \equiv \frac{x}{2 x - 1}$, we get the mapping
\begin{equation}
x \in \left[0,\frac12 - \frac{\epsilon}{4}\right] \cup \left[\frac12 + \frac{\epsilon}{4}, 1\right] \Rightarrow \hat{x} \in \left[0, \frac12 - \frac{1}{\epsilon}\right] \cup \left[\frac12 + \frac{1}{\epsilon}, 1\right]\,,
\end{equation}
with $\infty$ in $\hat{x}$ corresponding to $\frac{1}{2}$ in $x$. 
Applying this change of variable to Eq.~\eqref{eq:PVintegral} yields:
\begin{equation}
\label{eq: Bdiffsign}
B(k^2,M_1,M_2) = \sqrt{\pi}\left(\int_0^{\frac12 - \frac{1}{\epsilon}} d\hat{x} \frac{i}{\sqrt{-\hat{x} (1-\hat{x})k^2 - M_1 \hat{x} - M_2 (1-\hat{x})}} + \int_{\frac12 + \frac{1}{\epsilon}}^{1} d\hat{x} \frac{1}{\sqrt{\hat{x} (1-\hat{x})k^2+ M_1 \hat{x} +M_2 (1-\hat{x})}} + \pi\right)\,,
\end{equation}
which is a very familiar integrand (see Eq.~\eqref{eq:Bsame}).

We can further simplify this by noticing that 
\begin{equation}
s(-1, \hat{x} (1-\hat{x})k^2+ M_1 \hat{x} +M_2 (1-\hat{x})) = 1\,,
\end{equation}
if $x < 0$, $\Im(M_1) > 0$, and $\Im(M_2) > 0$ (because the second argument of $s$ always has a negative imaginary part in this case).
Therefore, we can simplify \eqref{eq: Bdiffsign}, obtaining
\begin{equation}
\label{eq: Bdiffsign2}
B(k^2,M_1,M_2) = \sqrt{\pi}\left(\int_0^{\frac12 - \frac{1}{\epsilon}}  \frac{d\hat{x}}{\sqrt{\hat{x} (1-\hat{x})k^2 + M_1 \hat{x} + M_2 (1-\hat{x})}} + \int_{\frac12 + \frac{1}{\epsilon}}^{1}  \frac{d\hat{x}}{\sqrt{\hat{x} (1-\hat{x})k^2+ M_1 \hat{x} + M_2 (1-\hat{x})}} + \pi \right)\,.
\end{equation}
Now both components have the same integrand. Let us define
\begin{align}
&I \equiv \int_0^{\frac12 - \frac{1}{\epsilon}}  \frac{d\hat{x}}{\sqrt{\hat{x} (1-\hat{x}) + m_1 \hat{x} + m_2 (1-\hat{x})}} \,,\\ 
&II \equiv \int_{\frac12 + \frac{1}{\epsilon}}^{1}  \frac{d\hat{x}}{\sqrt{\hat{x} (1-\hat{x}) + m_1 \hat{x} + m_2 (1-\hat{x})}}\,.
\end{align}
As written above in Eq.~\eqref{eq:Bfinalsame}, the antiderivative corresponding to $I$ and $II$ is
\begin{equation}
G(x) = i \log \left(2 \sqrt{x (1-x) + m_1 x + m_2 (1-x)}+i (m_1-m_2-2 x+1)\right)\,,
\end{equation}
where we have dropped the hat for clarity. Therefore, we have for $I + II$:
\begin{equation}
\label{eq:Bdiffsign3}
\begin{split}
&I + II = G(1) - G(0) + i \lim_{\epsilon \to 0} \left(\log \left(\frac{1}{4} i \epsilon  \Delta(m_1, m_2) \right) - \log \left(-\frac{1}{4} i \epsilon  \Delta(m_1, m_2) \right)\right)\,,
\end{split}
\end{equation}
where
\begin{equation}
\Delta(m_1, m_2) = m_1^2-2 m_1 m_2+2 m_1+m_2^2+2 m_2+1\,.
\end{equation}
Simplifying Eq.~\eqref{eq:Bdiffsign3} yields
\begin{equation}
\label{eq:Bdiffsign4}
\begin{split}
I + II &= G(1) - G(0) + i \left(- \log(-1) -\eta(-1,\frac{1}{4} i \epsilon  \Delta(m_1, m_2)) \right)\\
&= G(1) - G(0) + \pi - i \, \eta(-1, i \Delta(m_1, m_2))  \,,
\end{split}
\end{equation}
so Eq.~\eqref{eq: Bdiffsign2} simplifies to
\begin{equation}
\label{eq:Bfinaldiff}
B(k^2,M_1,M_2) = \sqrt{\pi}\left[G(1) - G(0) + 2\pi - i \, \eta(-1, i \Delta(m_1, m_2)) \right]\,.
\end{equation}
Note that $i \, \eta$ is either $2\pi$, 0, or $-2\pi$. This expression is almost identical to Eq.~\eqref{eq:Bfinalsame}.


Of course we have to take into account the branch cut crossings that can occur in the integration region. We will deal with them later.

\section{Triangle integral}
\section{Triangle integral with Schwinger parameters}
Let us compute the triangle integral,
\begin{align}
  T(k^2_1, k^2_2, M_1, M_2, M_3) = \int \frac{d^3q}{\pi^{3/2}}\frac{1}{((\vec{k}_1 - \vec{q})^2 + M_1) (q^2 + M_2)( (\vec{k}_2 +
\vec{q})^2 + M_3)}
\end{align}
in a similar way to the bubble integral. 

First let us consider the case of all three masses $M_1, M_2$ and $M_3$ having positive imaginary parts. Using Schwinger parameterization,
the momentum integrarl becomes,
\begin{align}
  T(\dots) = i\int \prod ds_i\int_q(1+i\epsilon_1)(1 + i\epsilon_2)(1 + i\epsilon_3)\exp\left[ i\left( (\vec{k}_1 - \vec{q})^2 + M_1
  \right)(1 + i\epsilon_1)s_1 + i(q^2 + M_2)(1 + i\epsilon_2)s_2 + i\left( (\vec{k}_2 + \vec{q})^2 + M_3 \right)(1 + i\epsilon_3)s_3
\right].
\end{align}
Expanding the exponential,
\begin{align}
  (1 + i\epsilon_1)s_1(k^2_1 + q^2 - 2k_1\cdot q + M_1) + (1 + i\epsilon_2)s_2(q^2 + M_2) + (1 + i\epsilon_3)s_3(k^2_2 + q^2 + 2k_2\cdot q +
  M_3)\\
  = q^2\left[ (1 + i\epsilon_1)s_1 + (1 + i\epsilon_2)s_2 + (1 + i\epsilon_3)s_3 \right] - 2q\cdot\left[ k_1(1 + i\epsilon_1)s_1 -
  k_2(1 + i\epsilon_3)s_3 \right] + (1 + i\epsilon_1)s_1(k^2_1 + M_1)\nonumber \\
  + (1 + i\epsilon_2)s_2M_2 + (1 + i\epsilon_3)s_3(k^2_2 + M_3)\\
  = F\left( q^2 - 2q\cdot\frac{(1 + i\epsilon_1)s_1k_2 - (1 + i\epsilon_3)s_3k_2}{F} \right) + (1 + i\epsilon_1)s_1(k^2_1 + M_1) +
  (1 + i\epsilon_2)s_2M_2 + (1 + i\epsilon_3)s_3(k^2_2 + M_3)\\
  = F\left( q - \frac{s_1k_1 - s_3k_2}{F} \right)^2 - \frac{(s_1k_1 - s_3k_2)^2}{F} + s_1k^2_1 + s_3k^2_2 + s_1M_1 + s_2M_2 + s_3M_3\\
  = F(\dots)^2 - \frac{s^2_1k^2_1 - 2s_1s_3k_1\cdot k_2+ s^2_3k^2_2}{F} + (s_1 + s_2 + s_3)(s_1k^2_1 + s_3k^2_2)/F + s_1M_1 + s_2M_2 + s_3M_3\\
  = F(\dots)^2  + \frac{s_1s_3k^2_3 + s_1s_2k^2_1 + s_2s_3k^2_2}{s_1 + s_2 + s_3} + s_1M_1 + s_2M_2 + s_3M_3
\end{align}
where we take the limit $\epsilon_i \rightarrow 0$ where no singularities appear and define $F = s_1 + s_2 + s_3$, and use $\vec{k}_1 +
\vec{k}_3 + \vec{k}_3 = 0$. Doing the Guassian integrarl in $q$, we obtain,
\begin{align}
  T = \frac{i}{(-i)^{3/2}}\int ds_1ds_2\frac{e^{iI}}{F^{3/2}}
\end{align}
where $I = (s_1s_2k^2_1 + s_3s_1k^2_3 + s_2s_3k^2_2)/F + s_1M_1 + s_2M_2 + s_3M_3$. Changing integration variables $s_1 = \tau x_1, s_2 =
\tau x_2, s_3 = \tau(1 - x_1 - x_2) = \tau x_3$,
\begin{align}
  T = \frac{1}{(-i)^{1/2}}\int dx_1dx_2 \tau^{1/2}e^{i\tau I}.
\end{align}
Doing the $\tau$ integral,
\begin{align}
  T = \frac{i\sqrt{\pi}}{2}\int dx_1 dx_2I^{-3/2}.
\end{align}
Next, we make another change of variables $x_2  = (1 - x_1)y, x_1 = x$, and do the indefinite integral in $y$,
\begin{align}
  T(x, y, k^2_1, k^2_2, k^2_3, M_1, M_2, M_3) = -\frac{\sqrt{\pi}}{2}\int dx \frac{\rho_N(y) x + c(y)}{\sqrt{\rho_S(y)\left( \gamma_R(y) - (x + x_0(y))^2
  \right)}\rho_D(y)\left( (x + x_1(y))^2 - \gamma_L(y) \right)}
\end{align}
where $\rho_N(y), c(y), \rho_S(y), \gamma_R(y), \rho_D(y), \gamma_L(y), x_0(y), x_1(y)$ are functions independent of the Feynman parameter
$x$. $T(\dots)$ is now integrable, but first we can put it in a simplified form,
\begin{align}
  T(\dots) &=  \int dx \frac{\sqrt{\pi}}{2}\frac{\rho_N}{\rho_D\sqrt{\rho_S(x-y_+)(x - y_-)}}\left( \frac{c + t_-}{(t_- - t_+)(x - t_-)} -
  \frac{c + t_+}{(t_- - t_+)(x - t_+)} \right) \vert^{y = 1}_{y = 0}\\
  &= \int dx \left( \frac{P(\rho_N, \rho_D,  t_{-}, c)}{\sqrt{\rho_S(x - y_+)(x - y_-)}(x - t_-)} - \frac{P(\rho_N, \rho_D,  t_{+}, c)}{\sqrt{\rho_S(x -
  y_+)(x - y_-)}(x - t_+)} \right)\vert^{y = 1}_{y = 0}\\
  &= P(\rho_N,\rho_D,t_-,c)F_{\rm int}(t_-, y_{\pm}) - P(\rho_N,\rho_D,t_+,c)F_{\rm int}(t_+, y_{\pm})\vert^{y = 1}_{y = 0}\vert^{x =
  1}_{x = 0}
\end{align}
where $y_{\pm} = \pm \sqrt{\gamma_R}-x_0$ and $t_{\pm} = \pm \sqrt{\gamma_L} - x_1$. The branch cuts of $F_{\rm int}$ are discussed in
section 5.1.

\subsection{Triangle for masses with positive real}
\textcolor{red}{Let us consider the case of $\left\{ \Re(M_1),\Re(M_2),\Re(M_3) \right\} > 0$. In this case we can choose the Schwinger parameterization,
\begin{align}
  \frac{1}{A} = \int^{\infty}_0ds \exp(-As)
\end{align}
for $\Re(A)>0$. In this case, the triangle integral can be written as,
\begin{align}
  T(k^2_1,k^2_2,M_1,M_2,M_3) = \int_{q}\int \prod ds_i \exp\left[ -\left((\vec{k}_1 - \vec{q})^2 + M_1\right)s_1 - (q^2 + M_2)s_2 - \left(
  (\vec{k}_2 - \vec{q})^2 + M_3 \right)s_3 \right].
\end{align}
Since if $\left\{ \Re(M_1),\Re(M_2),\Re(M_3) \right\}> 0$, $\Re(A)>0$. Expanding the exponential,
\begin{align}
  -(s_1 + s_2 + s_3)\left( q^2 - 2q\cdot\frac{s_1k_1 - s_3k_2}{s_1 + s_2 + s_3} \right) - s_1k^2_1 - s_1M_1 - s_2M_2 - s_3k^2_2 - s_3M_3\\
  = -(s_1 + s_2 + s_3)\left(q - \frac{s_1k_1 - s_3k_2}{s_1 + s_2 + s_3}\right)^2 + \frac{(s_1k_1 - s_3k_2)^2}{s_1 + s_2 + s_3} -s_1k^2_1 - s_3k^2_2 -
  s_1M_1 - s_2M_2 - s_3M_3\\
  = -(s_1 + s_2 + s_3)\left( q - \frac{s_1k_1 - s_3k_2}{s_1 + s_2 + s_3} \right)^2 - \frac{s_1s_3k^2_3 + s_1s_2k^2_1 +
  s_2s_3k^2_2}{s_1 + s_2 + s_3} - s_1M_1 - s_2M_2 - s_3M_3
\end{align}
We can perform the Gaussian integral as $s_1 + s_2 + s_3 > 0$. After the Gaussian integral in $q$, we make the change of variables $s_1 = \tau x_1, s_2 = \tau x_2, s_3 = \tau(1 - x_1 - x_2) = \tau x_3$,
\begin{align}
  \int dx_1dx_2\tau^2\frac{\exp\left[ -\tau\frac{x_1x_3k^2_3 + x_1x_2k^2_1 + x_2x_3k^2_2}{x_1 + x_2 + x_3} - \tau x_1M_1 - \tau x_2M_2 - \tau x_3M_3
  \right]}{(\tau x_1 + \tau x_2 + \tau x_3)^{3/2}}\\
  = \frac{\sqrt{\pi}}{2}\int dx_1dx_2 \frac{1}{\left(x_1(1 - x_1 - x_2)k^2_3 + x_1x_2k^2_1 + x_2(1 - x_1 - x_2)k^2_2 + x_1M_1 + x_2M_2 +
  (1 - x_1 - x_2)M_3\right)^{3/2}}
\end{align}
where in the last line we performed the $\tau$ integration. We now arrive at the same integral as the case when $M_1, M_2$ and $M_3$ having
positive imaginary parts.}

Let us consider the case of two masses $M_1$ and $M_2$ having positive imaginary parts and $M_3$ having a negative imaginary part. 
\begin{align}
  T(\dots) &= -i\int \prod ds_i \int_q (1 + i\epsilon_1)(1 + i\epsilon_2)(1 - i\epsilon_3) \times \nonumber \\
  &\exp\left[ i\left( (k_1 - q)^2 + M_1\right)(1 + i\epsilon_1)s_1 + i\left( q^2 + M_2 \right)(1 + i\epsilon_2)s_2 - i\left( (k_2 + q)^2 +
  M_3 \right)(1 - i\epsilon_3)s_3 \right]
\end{align}
Expanding the exponential,
\begin{align}
  (1 + i\epsilon_1)s_1(k^2_1 + q^2 - 2k_1\cdot q + M_1) + (1 + i\epsilon_2)s_2(q^2 + M_2) - (1 - i\epsilon_3)s_3(k^2_2 + q^2 + 2k_2\cdot q +
  M_3)\\
  = q^2\left[ (1 + i\epsilon_1)s_1 + (1 + i\epsilon_2)s_2 - (1 - i\epsilon_3)s_3 \right] - 2q\cdot\left[k_1(1 + i\epsilon_1)s_1 +
  k_2(1 - i\epsilon_3)s_3  \right] + (1 + i\epsilon_1)s_1(k^2_1 + M_1) \nonumber \\
  + (1 + i\epsilon_2)s_2M_2 - (1-i\epsilon_3)s_3(k^2_2 + M_3)\\
  = F \left( q^2 - 2q\cdot \frac{(1 + i\epsilon_1)s_1k_1 + (1 - i\epsilon_3)s_3k_2}{F}\right) + (1 + i\epsilon_1)s_1(k^2_1+M_1) + (1 +
  i\epsilon_2)s_2M_2 - (1 - i\epsilon_3)s_3(k^2_2+M_3)\\
  =F\left(q - \frac{s_1k_1 + s_3k_2}{F}\right)^2 - \frac{\left[s_1k_1 + s_3k_2 \right]^2}{F} + s_1k^2_1 - s_3k^2_2 + s_1M_1 + s_2M_2 -
  s_3M_3\\
  = F(\dots)^2 - \frac{s^2_1k^2_1 + 2s_1s_3k_1\cdot k_2 + s^2_3k^2_2}{F} + (s_1 + s_2 - s_3)(s_1k^2_1 - s_3k^2_2)/F + s_1M_1 + s_2M_2 -
  s_3M_3\\
  = F(\dots)^2 - \frac{2s_1s_3k_1\cdot k_2 + s_1s_3k^2_2 - s_1s_2k^2_1 + s_2s_3k^2_2 + s_1s_3k^2_1}{F} + s_1M_1 + s_2M_2 - s_3M_3\\
  = F(\dots)^2 + \frac{s_1s_2k^2_1 - s_1s_3k^2_3 - s_2s_3k^2_2}{F} + s_1M_1 + s_2M_2 - s_3M_3
\end{align}
where we take the limit $\epsilon_i \rightarrow 0$ where no singularities appear and redefine $F = s_1 + s_2 - s_3 + i\tilde{\epsilon}$.
After the Gaussian integral we obtain,
\begin{align}
  T = \frac{-i}{(-i)^{3/2}}\int ds_1ds_2 \frac{1}{F^{3/2}}e^{iI}
\end{align}
where $I = (s_1s_2k^2_1 - s_3s_1k^2_3 - s_2s_3k^2_2)/F + s_1M_1 + s_2M_2 - s_3M_3$. Change of variables $s_1 = \tau x_1, s_2 = \tau x_2, s_3
= \tau (1 - x_1 - x_2) = \tau x_3$,
\begin{align}
  T = \frac{1}{(-i)^{1/2}}\int dx_1dx_2d\tau \frac{\tau^{1/2}}{\left(2(x_1 + x_2) - 1 + i\tilde{\epsilon}\right)^{3/2}}e^{i\tau
  \tilde{I}}
\end{align}
where $\tilde{I} = \frac{x_1x_2k^2_1 - x_2x_3k^2_2 - x_1x_3k^2_3}{2(x_1 + x_2) - 1 + i\tilde{\epsilon}} + x_1M_1 + x_2M_2 - x_3M_3$. Doing
the $\tau$ integral yields,
\begin{align}
  T = -i\frac{\sqrt{\pi}}{2}\int^1_0dx_1dx_2 dx_3\delta(1-\sum x_i)\left( x_1 + x_2 - x_3 + i\tilde{\epsilon} \right)^{-3/2}\tilde{I}^{-3/2}
\end{align}

Again evaluating this integral with the principal value,
\begin{align}
  T = -\frac{\sqrt{\pi}}{2}\int dx_1dx_2dx_3\delta(1-\sum x_i)\left( P.V. \frac{1}{x_1 + x_2 - x_3} - \frac{i\pi}{2}\delta(x_2 + x_1 -
  \frac{1}{2}) \right)(x_1 + x_2 - x_3 + i\epsilon)^{-1/2}\tilde{I}^{-3/2}
\end{align}
however, the second term not involving the principal value evaluates to 0 so we are left with,
\begin{equation}
  \label{eq:trimxy}
  T = -\frac{\sqrt{\pi}}{2}P.V.\int dx_1dx_2dx_3 \delta(1-\sum x_i)\frac{1}{x_1 + x_2 - x_3}\frac{1}{\sqrt{x_1 + x_2 - x_3 + i\epsilon}}
  \tilde{I}^{-3/2}.
\end{equation}
We can then make the following substitutions, $x = \frac{x_1}{2(x_1 + x_2) - 1}, y = \frac{x_2}{2(x_1 + x_2) - 1}$ to obtain the
following integral,
\begin{align}
  T &=  \frac{\sqrt{\pi}}{2}P.V.\int dydx (2(x + y) - 1)^{3/2}\left( \frac{M_1x + k^2_2y + M_2y + k^2_1xy - k^2_2xy - k^2_2y^2 +
  M_3(1-x-y)+k^2_3x(1-x-y)}{2(x + y)-1} \right)^{-3/2}\\
  &= \frac{\sqrt{\pi}}{2}\int^1_0dy\left( \int^{1/2 - y - 1/\epsilon}_0dx + \int ^{1-y}_{1/2 - y + 1/\epsilon}dx \right)\frac{1}{(k^2_1xy +
  k^2_2(1-x-y)y + k^2_3x(1-x-y) + M_1x + M_2y + M_3(1-x-y))^{3/2}}\\
\end{align}
where we have the mapping,
\begin{align}
  x_1 \in \left[ 0, \frac{1}{2} - x_2 - \frac{\epsilon}{4} \right]\cup\left[ \frac{1}{2} - x_2 + \frac{\epsilon}{4}, 1 - x_2 \right] \Rightarrow
  x \in \left[ 0, \frac{1}{2} - y - \frac{1}{\epsilon} \right]\cup \left[ \frac{1}{2} - y + \frac{1}{\epsilon}, 1-y \right],
\end{align}
with $\infty$ in $y$ corresponding to $1/2 - x_2$ in $x_1$. The integrand can now be cast into the form,
\begin{align}
  T = \frac{\sqrt{\pi}}{2}\int^1_0dy\left( \int^{-\infty}_0dx + \int ^{1-y}_{\infty}dx \right)\frac{1}{\left( -\rho(x +z_1)(x + z_0)
  \right)^{3/2}}
\end{align}
where $\rho = k^2_3$ and $z_{0,1}$ are the solutions to the quadratic equation,
\begin{align}
  z_0 &= \frac{ -1 - M_1 + M_3 - k^2_1y + k^2_2y - k^2_3y + \sqrt{(-k^2_3 -M_1+M_3-k^2_1y+k^2_2y + k^2_3y)^2 -
  4k^2_3(-M_3-k^2_2y-M_2y+M_3y+k^2_2y^2)}}{2}\\
  z_1 &= \frac{ -1 - M_1 + M_3 - k^2_1y + k^2_2y - k^2_3y - \sqrt{(-k^2_3 - M_1 + M_3 - k^2_1y + k^2_2y + k^2_3y)^2 - 4k^2_3(-M_3 - k^2_2y - M_2y +
  M_3y + k^2_2y^2)}}{2}.
\end{align}
To analyze the branch points and branch cuts of the integrand, we consider,
\begin{align}
  T &=  \frac{\sqrt{\pi}}{2}\int^1_0dy\left( \int^{-\infty}_0dx + \int^{1-y}_{\infty}dx
  \right)\frac{s(-(x+z_1),x+z_0)}{(-(x+z_1))^{3/2}(x+z_0)^{3/2}\rho^{3/2}}\\
\end{align}
We may then choose the branch cuts as rays beginning at $z_0$ and $z_1$ and extending to $+\infty + i\Im(z_0)$ and $-\infty + i\Im(z_1)$
respectively. In the case where the two branch cuts of the integrand do not lie on the real line, the integrand is analytic on the real
axis, 
\begin{align}
  I &=  \int^{\infty}_{-\infty}dx\frac{1}{\left( \sqrt{\rho}\sqrt{-(x + z_1)}\sqrt{x + z_0} \right)^3} \\
  &= 0
\end{align}
In the case where either one or both branch cuts lie on the real axis, we can take,
\begin{align}
  I &=  \lim_{\epsilon \rightarrow 0}\int^{\infty + i\epsilon}_{-\infty + i\epsilon}dx \frac{1}{\left( \sqrt{\rho}\sqrt{-(x + z_1)}
  \sqrt{x + z_0} \right)^3}\\
  &= 0.
\end{align}
Thus, in both cases $I = 0$, and we have,
\begin{align}
  \int^{\infty}_{-\infty}dx\frac{1}{\left( \sqrt{\rho}\sqrt{-(x + z_1)}\sqrt{x + z_0} \right)^3} &=  0\\
  \int^{1-y}_{0}dx\frac{1}{\left( \sqrt{\rho}\sqrt{-(x + z_1)}\sqrt{x + z_0} \right)^3} &=  \left( \int^{-\infty}_0dx +  \int^{1-y}_{\infty}
  dx\right)\frac{1}{\left( \sqrt{\rho}\sqrt{-(x + z_1)}\sqrt{x + z_0} \right)^3}
\end{align}
and hence,
\begin{align}
  T =  \frac{\sqrt{\pi}}{2}\int^{1}_0\int^{1-y}_{0}dydx\frac{ s(-z_1(y),z_0(y))s(-1,z_1(y))}{\left( xyk^2_1+k^2_2(1 - x - y)y + k^2_3(1 - x - y)x + M_1x + M_2y + M_3(1 - x - y)
  \right)^{3/2}}.
\end{align}
Thus, in the case where $M_1$ and $M_1$ have positive imaginary mass and $M_3$ negative, we retrieve the same integral as before, and we can
identify $z_1 = -y_1, z_0 = -y2$.  
\subsection{Prefactor in antiderivative}

The expression for $F_{\rm int}$ is
\begin{equation}
F_{\rm int}(x,y_1,y_2,x_0) = P(a,x,y_1,y_2)\frac{2 \arctan\left(\frac{\sqrt{x-y_1} \sqrt{x_0-y_2}}{\sqrt{x-y_2} \sqrt{y_1-x_0}}\right)}{\sqrt{y_1 - x_0}\sqrt{x_0 - y_2}}\,,
\end{equation}
where the function $P$ is given by:
\begin{equation}
P(\rho,x,y_1,y_2)=\frac{\sqrt{x-y_1} \sqrt{x-y_2}}{\sqrt{-\rho (x-y_1) (x-y_2)}}\,.
\end{equation}
We can further simply $P$,
\begin{align}
  P(\rho,x,y_1,y_1) = \frac{1}{i\sqrt{\rho}}s(y_1,-y_2)s(-1,-y_1)
\end{align}

We also know that $F_{\rm int}$ is symmetric under $y_1 \Leftrightarrow y_2$.


We only need to calculate this expression for $x \to 0$, keeping $x>0$. Also, $a$ is a negative real number. Let us use our product rule for square roots (defined in Section~\ref{sec:prelim}) to simplify this expression. In general we can just calculate this directly, but we may run into trouble when $y_1$ or $y_2$ vanish. Let us show how to deal with that.
Take $y_1=0$. Then
\begin{equation}
P(a,x,y_1=0,y_2) = \frac{\sqrt{x}\sqrt{x-y_2}}{\sqrt{a x (x-y_2)}} = \frac{\sqrt{x}\sqrt{x-y_2}}{\sqrt{|a|}\sqrt{x}\sqrt{-(x-y_2)}} =  \frac{\sqrt{-y_2}}{\sqrt{|a|}\sqrt{y_2}}\,,
\end{equation}
and similarly
\begin{equation}
P(a,x,y_1,y_2=0) =  \frac{\sqrt{-y_1}}{\sqrt{|a|}\sqrt{y_1}}\,,
\end{equation}
and in the case where both $y_1$ and $y_2$ vanish:
\begin{equation}
P(a,x,y_1=0,y_2=0) = \frac{x}{\sqrt{a x^2}} = \frac{1}{\sqrt{a}} \,.
\end{equation}
Note that these results do not depend on $x$.

\subsection{Limiting values for $\arctan$}

Now that $P$ is well defined, let us look at $\tilde{F}_{\rm int} \equiv \frac{2 \arctan\left(\frac{\sqrt{x-y_1} \sqrt{x_0-y_2}}{\sqrt{x-y_2} \sqrt{y_1-x_0}}\right)}{\sqrt{y_1 - x_0}\sqrt{x_0 - y_2}}$. The issues that we are focusing on are:
\begin{enumerate}
\item When the result is indeterminate;
\item When we cross a branch cut.
\end{enumerate}

The result can be indeterminate in 2 situations: when $x=y_2$ and when $x_0 = y_2$.
Regarding the first case, we know that $x$ runs from 0 to 1, and we know that, for a general complex $z$:
\begin{align}
&\lim_{X \to 0^+} \arctan\left(\frac{z}{X}\right) = \frac{\sqrt{z^2}\pi}{2 z} \label{eq:limit1}\\
&\lim_{X \to 0^+} \arctan\left(\frac{z}{i X}\right) = i \frac{\sqrt{-z^2}\pi}{2 z}\,. \label{eq:limit2}
\end{align} 
The problematic values of $y_2$ are 0 and 1 because they are the values at which we calculate the antiderivative. 

When $y_2 = 0$, we have for a general function $f$, $\lim_{x \to 0^+} f(\sqrt{x-y_2}) = \lim_{X \to 0^+} f(X)$. 
When $y_2 = 1$, we have $\lim_{x \to 1^-} f(\sqrt{x-y_2}) = \lim_{X \to 0^+} f(i X)$. 
Therefore when $y_2 = 0$ we use Eq.~\eqref{eq:limit1} with $z = \frac{\sqrt{-y_1} \sqrt{x_0}}{\sqrt{y_1-x_0}}$, giving 
\begin{equation}
\tilde{F}_{\rm int} = \sqrt{\left(\frac{\sqrt{-y_1} \sqrt{x_0}}{\sqrt{y_1-x_0}}\right)^2}\frac{\pi}{2}\frac{\sqrt{y_1-x_0}}{\sqrt{-y_1} \sqrt{x_0}}= \sqrt{\frac{-y_1\, x_0}{y_1-x_0}}\frac{\pi}{2}\frac{\sqrt{y_1-x_0}}{\sqrt{-y_1} \sqrt{x_0}}\,,
\end{equation}
and this last expression can be simplified using the generalized product rule for square roots.



 and when $y_2 = 1$ we use Eq.~\eqref{eq:limit2}, with the same $z$.

When $x_0 = y_2$ we just use the Taylor approximation for $\arctan$ for a small argument $z$: $\arctan(z) \approx z$.



\end{document}

%        File: !comp!expand("%")!comp!
%     Created: !comp!strftime("%a %b %d %I:00 %p %Y ").substitute(strftime('%Z'), '\<\(\w\)\(\w*\)\>\(\W\|$\)', '\1', 'g')!comp!
% Last Change: !comp!strftime("%a %b %d %I:00 %p %Y ").substitute(strftime('%Z'), '\<\(\w\)\(\w*\)\>\(\W\|$\)', '\1', 'g')!comp!
%
\documentclass[twoside]{article}
\usepackage[tmargin=2.5cm,bmargin=2.5cm,lmargin=2cm,rmargin=2cm]{geometry}
\usepackage{graphicx}
\usepackage{caption}
\usepackage{enumerate}
\usepackage{framed}
\usepackage{amssymb}
\usepackage[svgnames]{xcolor}
\usepackage{comment}
\usepackage[normalem]{ulem}
\usepackage{amsmath}
\usepackage{fancyhdr}
\usepackage{hyperref}
\usepackage{physics}
\usepackage{siunitx}
\usepackage{float}
\pagestyle{fancy}
\fancyhf{}
\lhead{}
\chead{}
\cfoot{\thepage}

\begin{document}

\section{The problem}
We are interested in two types of integrals, which we call the bubble integral and the triangle integral. The bubble integral has the form,
\begin{align}
  \int \frac{d^3q}{(2\pi)^{3/2}}\frac{1}{(q^2 + M_1)( (k-q)^2 + M_2)}
\end{align}
and the triangle integral has the form,
\begin{align}
  \int \frac{d^3q}{(2\pi)^{3/2}}\frac{1}{(q^2 + M_1)( (k_1-q)^2 + M_2)( (k_2 + q)^2 + M_3)}.
\end{align}
To compute these integrals, we use Feynman parameterization and dim reg. When the masses are real $M_1,M_2$ and $M_3$ are real, we do not
have to worry about branch cuts and analytic continuation of the above integral in the complex plane. However, we will need to be more
careful in the case of complex masses. 

One is allowed to introduce Feynman parameterization even with complex masses as this does not introduce extra branch cuts to the integrand.

\section{Bubble integral}
Let us focus on the bubble integral. We have,
\begin{align}
  \int_{q} \frac{1}{(q^2 + M_1)( (k-q)^2 + M_2)} &=  \int_{q}dx\frac{1}{(q^2(1-x) + M_1(1-x) + (k-q)^2x + M_2x)^2}\\
  &= \int dx \int_{l}\frac{1}{(l^2 + D)^2}
\end{align}
where $l = q + xk$ and $D = x(1-x)k^2 + M_2 + x(M_1-M_2)$. Since $x,k$ and $q$ are all real, $l$ remains real. We can anlytically continue
the integration of $l$ to the complex plane and take the usual semicircle contour since the integrand vanishes at $\lvert l \rvert
\rightarrow \infty$. The integrand has a second order pole at $l = \pm i\sqrt{D}$. Since the poles will always be distributed
such that one exists in the upper half plane and the other in the lower half plane, it does not matter if we choose to close the contour in
the upper or lower half of the plane. After computing the residue of the pole we get the usual dim reg result,
\begin{align}
  \int dx \int_{l} \frac{1}{(l^2 + D)^2} = \int dx \frac{\sqrt{\pi}}{D^{1/2}}
\end{align}
However, there is one subtlety, which is that we need to check that the poles never lie on the real axis. This can easily be shown.
In order for a pole to lie on the real axis, we would require $\Re(\sqrt{D}) = 0$ and $\Im(\sqrt{D})\neq 0$. However, $D$ has the form,
\begin{align}
  D = x(1-x)k^2 + M_2 + x(M_1 - M_2)
\end{align}
The first term will always be equal or greater than 0, and since $M_2$ always has a real component for non-zero masses, $D$ will never have
0 real component to it, and thus we will never cross any poles as we integrate over the real axis. Therefore,
\begin{align}
  \sqrt{\pi}\int dx D^{-1/2} = i\log\left( 2\sqrt{D} + i(m_1 + m_2 - 2x + 1) \right).
\end{align}
Where $m_1 = M_1/k^2$ and $m_2 = M_2/k^2$. 

\section{Triangle integral}
The triangle integral has the form,
\begin{align}
  \int \frac{d^3q}{(2\pi)^{3/2}}\frac{1}{(q^2 + M_1)( (k_1 - q)^2 + M_2)( (k_2 + q)^2 + M_3)}.
\end{align}
After introducing Feynman parameters we have an integral of the form,
\begin{align}
  \int dx dy \int_l \frac{1}{(l^2 + D)^3}
\end{align}
again $l$ is real and we can analytically continue to the complex plane and take the contour integral. The integrand contains a third order
pole and we can compute the residue the same way to get the dim reg result. The poles are again at $l = \pm i \sqrt{D}$ and since the masses
$M_1, M_2$ and $M_3$ always have non-zero real components, we never have poles on the real axis, and thus the dim reg result is always true. 
\end{document}


